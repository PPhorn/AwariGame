\documentclass{report}
\usepackage{tikz}
\usepackage[margin=2.5cm]{geometry}
\usepackage{hyperref}
\usepackage{graphicx}
\graphicspath{{figures/}{anotherFigureDirectory/}}
\graphicspath{ {./images/} }
\usepackage{listings}
\usepackage{wrapfig}
\usepackage{color}
\definecolor{bluekeywords}{rgb}{0.13,0.13,1}
\definecolor{greencomments}{rgb}{0,0.5,0}
\definecolor{turqusnumbers}{rgb}{0.17,0.57,0.69}
\definecolor{redstrings}{rgb}{0.5,0,0}
\definecolor{gray}{rgb}{0.13,0.13,0.13}
\lstdefinelanguage{FSharp}
                {morekeywords={let, new, match, with, rec, open, module,
                namespace, type, of, member, and, for, in, do, begin, end, fun,
                function, try, mutable, if, then, else},
                keywordstyle=\color{bluekeywords},
                sensitive=false,
                numbers=left,  % where to put the line-numbers;(none, left, right)
                numberstyle=\tiny\color{gray},
                morecomment=[l][\color{greencomments}]{///},
                morecomment=[l][\color{greencomments}]{//},
                morecomment=[s][\color{greencomments}]{{(*}{*)}},
                morestring=[b]",
                showstringspaces=false,
                stringstyle=\color{redstrings}
                }

\title{PoP - Ugeopgave 7}
\author{Christoffer, Inge og Pernille}
\date{\today}

\begin{document}
\maketitle
\tikzstyle{block} = [rectangle, draw, fill=blue!20, text centered,
    rounded corners, minimum height=2.5em]
\tikzstyle{cloud} = [rectangle, draw, fill=white, text centered,
    rounded corners, minimum height = 2em]
\tikzstyle{line} = [draw, -latex]

\section*{Preface}
As a part of the Programming and Problem Solving course, we,
three Computer Science students at Copenhagen University, build the game Awari.



\subsection*{getMove}
{\it This function gets a player's next move.}
\lstinputlisting[language=FSharp, firstline=176, lastline=195]{AwariLib.fs}
The function uses \texttt{System.Console.Readline} to get a keyboard input and then translates this to a pit number. It is only possible for the player to choose between 1-6. If another key is entered, the player is asked to try again. The function matches the player with the input to make sure to return a valid pit in consistence with the player.
\\

\subsection*{checkOpp}
{\it This function checks whether there are any beans in the opposite pit of the final pit.}
\lstinputlisting[language=FSharp, firstline=208, lastline=213]{AwariLib.fs}
\texttt{checkOpp} is of type boolean and returns true if there are beans in the opposite pit from the final pit. If the opposite pit is any of the home pits, or if the opposite pit contains zero beans, then it returns false. \texttt{checkOpp} is used by the function \texttt{distribute} to take the beans from the opposite pit to the home pit if the final pit is empty.

\subsection*{finalPitPlayer}
{\it This function returns the player of the final pit.}
\lstinputlisting[language=FSharp, firstline=223, lastline=226]{AwariLib.fs}
If the final pit is less or equal to 6, \texttt{finalPitPlayer} returns player 1. Else it returns player 2. The function is used in the \texttt{distribute} function.

\subsection*{distribute}
{\it This function distributes the beans from the chosen pit counter clockwise}
\lstinputlisting[language=FSharp, firstline=239, lastline=262]{AwariLib.fs}
\texttt{distribute} is one of the key function in the game as already described. It sets a variable k equivalent to the number of beans in the chosen pit i. It then uses a while loop to distribute the number of beans k in the following pits, j, by adding one to b.[j] while taking one from k by each distribution. It then updates the finalPit to be j and checks whether there are any beans in the pit opposite the end pit. If so, the function adds them to the home pit and then updates the final pit and the opposite pit to be zero. At the end it returns a tuple containing of the updated board, the final pit player and the final pit. \texttt{distribute} is called by the \texttt{turn} function.

\subsection*{turn}
{\it The function calls the turn of a player}
\lstinputlisting[language=FSharp, firstline=279, lastline=294]{AwariLib.fs}
\texttt{turn} is another key function of the programme. It was handed  to us from the beginning, but we have altered it slightly. It contains a recursive function \texttt{repeat} that gets the player, the board and the entered 	pit number from \texttt{getMove} to evaluate who's turn it is and returns the new board. It calls for \texttt{printBoard} to print the board and then it call for \texttt{playerAsString} to let the players know who's turn it is. 

Then it makes three variables, newB, finalPitsPlayer and finalPit and updates them via a call to \texttt{distribute} to the board, the player and i. i is a pit number retrieved from a call to \texttt{getMove}. 
Then the function checks if the final pit is not the player's home (it calls for \texttt{isHome}), or if the game is over (it calls for \texttt{isGameOver}) and then it returns a new board, newB. Else the \texttt{repeat} function calls itself as it is the same player again. Then it calls itself to the new player.


\subsection*{play}
{\it The only function called in the application file}
\lstinputlisting[language=FSharp, firstline=306, lastline=324]{AwariLib.fs}
The \texttt{play} function is the only function that is called in the application file. It was also a piece of code that was handed to us but we have altered it some.
At first, it asks if the game is over as it calls \texttt{isGameOver}. If true, it prints the board and announces the winner. We have added the winner announcement to the function. If the game is not over jet, it returns a new board newB that calls for \texttt{turn}, and the next player nextP. Then it calls itself recursively to newB and newP.