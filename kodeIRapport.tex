\documentclass[a4paper]{report}
\usepackage{tikz}
\usepackage[margin=2.5cm]{geometry}
\usepackage{hyperref}
\usepackage{graphicx}
\graphicspath{{figures/}{anotherFigureDirectory/}}

% Package for typesetting programs. Listings does not support fsharp, but a
% little modification goes a long way
\usepackage{listings}
\usepackage{color}
\definecolor{bluekeywords}{rgb}{0.13,0.13,1}
\definecolor{greencomments}{rgb}{0,0.5,0}
\definecolor{turqusnumbers}{rgb}{0.17,0.57,0.69}
\definecolor{redstrings}{rgb}{0.5,0,0}
\definecolor{gray}{rgb}{0.13,0.13,0.13}
\lstdefinelanguage{FSharp}
                {morekeywords={let, new, match, with, rec, open, module,
                namespace, type, of, member, and, for, in, do, begin, end, fun,
                function, try, mutable, if, then, else},
                keywordstyle=\color{bluekeywords},
                sensitive=false,
                numbers=left,  % where to put the line-numbers;(none, left, right)
                numberstyle=\tiny\color{gray},
                morecomment=[l][\color{greencomments}]{///},
                morecomment=[l][\color{greencomments}]{//},
                morecomment=[s][\color{greencomments}]{{(*}{*)}},
                morestring=[b]",
                showstringspaces=false,
                stringstyle=\color{redstrings}
                }

\title{PoP - Ugeopgave 7}
\author{Christoffer, Inge og Pernille}
\date{\today}

\begin{document}
\maketitle
\tikzstyle{block} = [rectangle, draw, fill=blue!20, text centered,
    rounded corners, minimum height=2.5em]
\tikzstyle{cloud} = [rectangle, draw, fill=white, text centered,
    rounded corners, minimum height = 2em]
\tikzstyle{line} = [draw, -latex]

\section*{Preface}
As a part of the Programming and Problem Solving course, we,
three Computer Science students at Copenhagen University, build the game Awari.

\section*{Awari}
Awari is an ancient two player game that resembles the beloved Kalaha. The main
objective of Awari is to capture the most beans. The game consists of a board
with 6 pits and one home pit for each player. Each of the 6 x 2 pits consists of
3 beans. The players must in turn take the amount of beans from a pit on his or
her side of the board and distribute them in the following pits. The game
continues until one of the two players has no beans left, and the winner is the
player who captured the most beans in his or her homepit.

\section*{Problem description}
We have implemented the Awari using the functional programming language F\#. We
were given two functions as a start, a turn and a play function, and a signiture
file with type indication for several minor functions. To be able to play the
game we have created functions to print the board, to distribute the beans, to
check for game over etc. All the functions will be elaborated in this rapport.

\section*{Problem analysis and design}
\subsection*{flow}
We designed our programme starting by focusing on the board. The board is the center of the game and thus the main part of takes the board as input.

\ref{fig:gameflow}).
\begin{figure}
\centering
\begin{tikzpicture} [node distance = 1.5 cm, auto]
		\node [cloud] (start) {The board is printed};
        \node [cloud, below of=start] (choise) {A Player choses a pit};
        \node [cloud, below of=choise] (distribute) {Beans are distributed};
        \node [cloud, below of=distribute] (eval) {The result is evaluated};
		\node [cloud, right of=eval, node distance = 4 cm]
            (newp) {The other players turn};
        \node [cloud, left of=eval, node distance = 4 cm]
            (end) {The game is over};
        % Arrows
        \path [line] (start) -- (choise);
        \path [line] (choise) -- (distribute);
        \path [line] (distribute) -- (eval);
        \path [line] (eval) -- (newp);
        \path [line] (newp) -- (start);
        \path [line] (eval) -- (end);

    \end{tikzpicture}
    \caption{Flow of Awari}
    \label{fig:gameflow}
\end{figure}
\newline






\lstset{language=FSharp}
\lstinputlisting{AwariLib.fs}
\end{document}
